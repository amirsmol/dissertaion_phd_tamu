
\chapter{\uppercase{Conclusion}}  
This study presents nonlinear and time-dependent electro-mechanical analyses of piezoelectric based materials and structures. Ceramics based piezoelectric materials, such as lead zirconate titanate (PZT), are considered, and their hysteretic behaviors under various magnitude of electric-field stimulus are studied. Phenomenological constitutive models have been formulation in order to capture minor and major loops of piezoelectric ceramics under cyclic electric field inputs. A quasi-linear viscoelastic (QLV) model is adopted in order to incorporate the time-dependent effect on the nonlinear electro-mechanical response of piezoelectric ceramics. These phenomenological models are implemented in continuum 3D finite elements, which are useful for analyzing behaviors of several piezoelectric structures and structural components under various boundary conditions. A time-integration algorithm, based on linearized predictor and corrector schemes, has been developed to solve for the nonlinear time-dependent electro-mechanical response at the material and structural levels. The nonlinear time-dependent electro-mechanical models have been validated with experimental data on PZT materials and structural components such as telescopic actuators and bimorph beams.

 
The integrated time-dependent electro-mechanical material model and finite element analysis is also used to study the overall electro-mechanical responses of active fiber composites (AFCs). AFCs comprise of long unidirectional piezoelectric fibers embedded in polymeric matrix and placed in between two electrode fingers on the top and bottom surfaces of the AFCs. Electric fields are prescribed through the electrode fingers in order to actuate the AFCs. AFCs form flexible piezoelectric ceramics based actuators suitable for morphing, bending, and/or twisting type of deformations. In this study, a unit-cell model of AFCs comprising of a segment of quarter fibers, epoxy matrix, and metallic electrode fingers is considered in order to reduce computational cost. The overall nonlinear electro-mechanical hysteretic responses of the unit-cell model are comparable to the ones of experimental data. The presented study on AFCs is useful in designing electro-active composites with various microstructural architectures and properties of constituents as it can give an estimate of the overall electro-mechanical responses of electro-active composites prior to manufacturing them.

 
Finally, the nonlinear time-dependent electro-mechanical constitutive model is integrated to 3D truss finite element and used to analyze shape changing behaviors of active truss systems. The active truss system consists of arrangements of truss elements, and some of the elements are integrated with active materials such as piezoelectric, in which shape changing in the entire truss systems is generated by activating some or all of the active elements. In this study, two types of truss arrangements are considered and deformed shapes of a truss structures are controlled by the application of electric stimuli. Parametric studies on the effect of time dependent and nonlinear constitutive equation in controlling the shapes of truss structures are conducted. 

 

