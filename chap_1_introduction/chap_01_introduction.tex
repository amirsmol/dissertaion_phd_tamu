%%%%%%%%%%%%%%%%%%%%%%%%%%%%%%%%%%%%%%%%%%%%%%%%%%%
%
%  New template code for TAMU Theses and Dissertations starting Fall 2012.  
%  For more info about this template or the 
%  TAMU LaTeX User's Group, see http://www.howdy.me/.
%
%  Author: Wendy Lynn Turner 
%	 Version 1.0 
%  Last updated 8/5/2012 
%  
%%%%%%%%%%%%%%%%%%%%%%%%%%%%%%%%%%%%%%%%%%%%%%%%%%%

%%%%%%%%%%%%%%%%%%%%%%%%%%%%%%%%%%%%%%%%%%%%%%%%%%%%%%%%%%%%%%%%%%%%%%
%%                           SECTION I
%%%%%%%%%%%%%%%%%%%%%%%%%%%%%%%%%%%%%%%%%%%%%%%%%%%%%%%%%%%%%%%%%%%%%
 

\pagestyle{plain} % No headers, just page numbers
\pagenumbering{arabic} % Arabic numerals
\setcounter{page}{1} 


\chapter{\uppercase {Introduction}}
Discovery of electro-mechanical coupling effect which was first seen on natural materials dates back to early days.
For example, Quartz and Rochelle salt are shown to exhibit electro-mechanical coupling behavior \cite{Lines1977}.
The ability of the materials to generate electric displacement when subjected to a mechanical force or displacement was called a direct effect.
The reason was, that the direct electro-mechanical coupling response was experimentally observed and quantified first.  
Curie brothers have studied the direct piezoelectric response \cite{Curie1882,Mould2007}
 which is attributed to the crystal structures of the materials.
The inverse piezoelectric effect is observed when an electric field input causes deformations in the materials.
This phenomena was also observed and mathematically quantified by Curie \cite{Curie1882,Mould2007}. 
The electro-mechanical coupling effect in ceramics is due to electric polarization.
Materials that have a spontaneous electric polarization that can be reversed by the application of an external electric field are known as ferroelectric \cite{Lines1977}.
The piezoelectric ceramics considered in this study are ferroelectric materials.
Piezoelectric materials first found their applications in sonar devices for emitting ultrasonic waves. 
Other applications of piezoelectric ceramics, are for sensors and actuators.
The sensing and actuation are limited to small ranges of motion.
Atomic force microscopy (AFM), and data reader and recorder from electro magnetic hard disk drives are just some of these applications. 
Recently, piezoelectric materials are being used in energy harvester devices \cite{erturk2011piezoelectric}. 


Piezoelectric ceramics can undergo relatively small displacements that limit their applications in mechanical systems.
There have been innovations in production of new amplification architectures for piezoelectric materials.
The telescopic actuators \cite{Alexander2003,Alexander}, piezo stack actuators \cite{Ardelean2004}, and active fiber composites \cite{atillah2014} are devices utilizing several actuation architectures.
There have been applications and designs of piezoelectric materials and devices in order to increase motion capability.
The variable geometry trusses are one of these possible applications that will be discussed and considered in this dissertation. 
Prior to designing active structures made of piezoelectric materials, there is a need to understand the electro-mechanical response of active materials.
This has been done through the development of constitutive models.
Consequently, analytical/numerical tools that allow for analyzing active structures are also needed. 

% Variable Geometry Trusses are discussed are alo discussed in this dissertation as another application of active materials.
% A VGT can be considered as a multi link robot.
% In this case VGTs are open-loop linkages with many degrees of freedoms. 
% In the analyses of VGT as a robot and controlling its movement we have to form solution method in two ways.
% First, to related strain to be produced in each truss element with overall configuration of truss.
% Next, to relate the desired shape of the truss for placement of specific point of the truss.
% For example if we need a beam like truss to have its endpoint in a specific point in space.
% The second problems will have many solution due to so many degrees of freedoms that VGT system offers.
% This research is offering a way to choose the best possible solution to second problem.  
% Also we offer a way for analyzing and simulating a truss system  required for the first problem.\\

This section presents a brief literature review.
The review focuses on the response of electro-active ceramic based materials,
 electro-mechanical constitutive models and numerical methods for analyzing electro-mechanical response of active materials and structures.
Motivation and research objectives are discussed at the end of this chapter. 
\\

\section{Literature Review} 
\subsection{Material Response}
In this study piezoelectric materials are being classified as hard and soft materials. 
Stiff electro-active materials are typically made of ceramics, such as lead zirconate titanate (PZT) and barium titanate ($BaTiO_3$).
The polymeric based materials, such as polyvinylidene fluoride (PVDF) and dielectric elastomer, are examples of soft electro-active materials. 
This literature review focuses on the piezoelectric ceramics based materials. 
Experimental studies on polarized piezoelectric materials, i.e. PZT, and piezoelectric devices \cite{Crawley1990,Ardelean2004,anderson1989piezoceramic} shows nonlinear \footnote{The nonlinear behavior is considered when the responses do not satisfy proportionality and superposition conditions.} electro-mechanical response, especially under large electric fields. 
Linear electro-mechanical constitutive equation is only applicable when piezoelectric materials are subjected to relatively small stimuli.
The large electric field applied will cause nonlinear response that does not follow proportionality and superposition of the input.
Therefore, linear electro-mechanical constitutive equations, can lead to substantial error when large electric fields are applied to electro active 
materials and devices \cite{Hall2001}. 
There have been several studies on understanding the nonlinear response of piezoelectric materials.
These studies are done by taking the coupling coefficient dependent on the electric fields \cite{Crawley1990}.
Another way to model the nonlinearity is by incorporating higher order electro-mechanical coupling effect in the constitutive equations. 
There are examples of nonlinear constitutive equation based on a higher order electro-mechanical coupling effect and their finite element implementation.
Some of these constitutive equations can be found in \cite{Bassiouny19881297,tiersten1993electroelastic,Maugin2010,Muliana2011a,Sohrabi2011},
 in which a linearized strain is considered.
Therefore, these models are applicable for materials undergoing small strains such as ceramics based piezoelectric materials.

Another aspect that has been observed in experimental studies of piezoelectric ceramics is their rate dependent responses \cite{Zhou2010,Zhou2006}. 
These effects are shown by time and field dependent piezoelectric coupling coefficients and hysteresis electro-mechanical coupling. 
Devices made of piezoelectric materials, such as actuators, sensors and energy harvesters often to operate under oscillating stimuli, 
at various frequencies. 
It has been observed that behaviors of piezoelectric materials, even under relatively low input regimes, are history dependent \cite{Crawley1990,anderson1989piezoceramic}. 
At higher amplitude of electro-mechanical stimuli, piezoelectric materials experience pronounced nonlinear time-dependent behaviors. 
It is then necessary to incorporate the nonlinear and time-dependent electro-mechanical coupling responses for analyzing performance of piezoelectric devices.
 
Several experimental studies have also been conducted on understanding response of ceramics based electro active materials 
under cyclic electric fields with amplitudes higher than the coercive electric field limits of the materials. 
Under such loading conditions, materials exhibit polarization switching.
This Nonlinear effect has been reported in Schmidt \cite{schmidtcoercive1981}, Gookin et al. \cite{gookinelectro-optic1984} and \cite{raey}.  
It was also shown that compressive stresses that are applied along the poling axis of the ferroelectric materials could induce depolarization of the poled ferroelectric materials \cite{lynch1996effect, chena1998, raey}. 
Fang and Li \cite{raey} experimentally studied changes in polarization and strain responses of a PZT specimen under a cyclic electric field input. 
After several cycles, the saturated polarization response converges to a constant value, which is slightly smaller than the one measured in the first cycle. 
An experimental study on a polarized PZT specimen under cyclic electric fields with the maximum amplitude of 85 percent of the coercive electric field of the PZT, reported by Crawley and Anderson \cite{Crawley1990}, 
also showed nonlinear hystersis electro-mechanical response.  
They observed that the effects of creep and loading rate on the piezoelectric constant were more significant at larger strains and lower frequencies. 
The electrical and mechanical responses of ferroelectric materials are time and frequency dependent.
The experimental evidence of this phenomena is shown by Fett and Thun \cite{fettdetermination1998}, Schaeufele and Hardtl \cite{schaufele1996ferroelastic}, Zhou and Kamlah \cite{zhoudetermination2005,Zhou2006}, Ben Atitallah et al. \cite{atillah2014}.
Among others, Zhou and Kamlah \cite{zhoudetermination2005,Zhou2006} showed the creep response in a soft PZT under static electric fields and compressive stresses, which were more pronounced at higher stresses and at electric fields near the coercive electric field.

There have been constitutive models developed to predict nonlinear electro-mechanical behaviors of electro-active ceramics undergoing polarization switching.
These models can be classified as phenomenological (macroscopic) model based on continuum mechanics approach and micro-mechanics based models.
In an analogy to rate-independent plasticity theory, the macroscopic constitutive models have been formulated for predicting polarization switching response in ferroelectric materials due to electric field inputs. 
In such cases, strains and electric displacements are additively decomposed into reversible and irreversible components.
The irreversible component incorporates the switching mechanism. 
Examples of the macroscopic models can be found in Bassiouny et al. \cite{Bassiouny1988,Bassiouny19881297}, Huang and Tiersten \cite{huang1998electroelastic,huang1998analytical}, Kamlah and Tsakmakis \cite{kamlahphenomenological1999}, Linnemann et al. \cite{Linnemann20091149}, Muliana \cite{Muliana2011}. 
Massalas et al. \cite{Massalas19941075} and Chen \cite{Chen201049} presented nonlinear electro-mechanical constitutive equations for materials with memory-dependent (viz. viscoelastic materials). 
They also incorporated the dissipation of energy due to the visco-elastic effect, which is converted into heat. 
It is known that the macroscopic response of materials depends strongly upon their micro-structural response, which occurs at various length scales. 
Microscopically motivated constitutive models that take into account polarization response of each crystal in predicting the overall nonlinear electro-mechanical response of ferroelectric materials can be found in Chen and Lynch \cite{chena1998}, Fan et al. \cite{raey}, Li and Weng \cite{Li19993493,Li200179}, Smith et al. \cite{Smith2003719,Smith200646}, Su and Landis \cite{Su2007280}.

Finite element (FE) method has been used for analyzing linear electro-mechanical responses of piezoelectric materials and structures. 
Allik and Hughes \cite{Allik1970} are among the first authors to present FE formulation of piezoelectric materials which is used in commercial FE codes. 
A review of various finite element formulations for simulating linear electro-mechanical responses of piezoelectric materials is presented in Benjeddou, \cite{Benjeddou2000}.
FE method has also been used for analyzing coupled  piezo-electro-hygro thermo viscoelasto-dynamic-problems \cite{Yi1999} which focuses on linear viscoelastic and field coupling response. 
FE analyses of nonlinear electro-mechanical and hysteresis polarization switching responses of structures consisting of conductive and ferroelectric materials are mainly available for time (rate)-independent behavior, e.g. Kamlah and Bohle \cite{Kamlah2001605}, Landis \cite{Landis2002}, Zeng et al. \cite{NME:NME556}, Li and Fang \cite{Li2004959}, Zhang et al. \cite{Zhang2005185}, Klinkel \cite{Klinkel20067197}, Wang and Kamlah \cite{0964-1726-18-10-104008}, Linnemann et al. \cite{Linnemann20091149}, Klinkel et al. \cite{Klinkel2006349}, and Muliana and Lin \cite{Muliana2011a}. 
Macroscopic constitutive models where considered in the above FE analyses.
Zeng et al. \cite{NME:NME556} presented incremental and iterative solutions for problems involving polarization switching due to high electric field and heat generation from the dissipation of energy during the domain reversal process. 
FE methods that also include the time (rate) - dependent effects are currently limited. 
Kim and Jiang \cite{Kim2002} presented FE algorithm for simulating macroscopic polarization and strain responses in ferroelectric materials undergoing domain switching. 
They also defined the functions for the rate of change of the mass fractions, which allow for incorporating rate-dependent loading.
\\

\subsection{Structural Behavior}

\subsubsection{Active Composite Beams}
Piezoelectric ceramics have been used in the form of layered composite beams. 
Their applications can be found for structural health monitoring, actuators and, recently energy harvesting devices. 
In the case of energy harvesting devices piezoelectric materials offer large power generating capacity compared to other energy harvesting sources \cite{erturk2011piezoelectric}. 
However, the small strains in piezoelectric ceramics limits their applications for motion generation. 
In order to overcome this restriction they are normally used in the form of stacked beams, bimorph and multi-layer beams which forms a composite structure. 
The piezo ceramics can be used in many different forms in order to magnify the displacement resulted from electromechanical coupling. 
When large forces are expected in the device the stacked piezoelectric is used,
 while, the bimorph configuration is preferred when the devices are intended to achieve large displacements.
Low and Guo \cite{low1995modeling} have formulated a mathematical model for a three layer composite piezoelectric bimorph beam.
A state variable was introduced in order to incorporate the hystersis response. 
An experimental set up for an actuator consists of four bimorph beams, one moving plate, two guides, and one base.
Each pair of bimorph beams was connected with two small hinges at each end so that
 the ends are free from moment and the only force acting on each end
 is the reacting force. 
They applied electric field with amplitude of 100 V on the beam with thickness of 0.0075 inch.
This would result in 2620 V/m electric field through the thickness of each piezo electric layer of the bimorph beam. 
They used the data from the test to calibrate the material parameters in their mathematical model.
A mathematical model for bending of piezoelectric composite layered beam was also developed by Raja et al. \cite{raja2004bending}.
A sandwich beams was considered in their research. 
They presented a closed form solution for bending of a composite piezoelectric bimorph beam, and
 they compared the result from their analytical solution with ABAQUS finite element analyses.
They also examine the effect of PZT-5H patches on bending of sandwich beam.  
They have shown that the transverse deflection of the beam predicted by their method correlates with results available in literature and also finite element analyses performed in ABAQUS.

Activation in shear mode for producing bending in a composite beam was studied by Kheidar et al. \cite{khdeir2001deflection}.
They compared the bending behaviors in beam by utilizing piezo electric patches for shear mode and extension mode.  
A first order beam theory was presented for the composite piezoelectric beam. 
The effect of transverse shear deformation was considered.
They have shown that using piezo electric patches in shear mode for actuation can lead to higher bending deflection. 
A comparison between extension and shear actuations was also considered by Benjeddou et al. \cite{benjeddou1999new}.
They analytically investigated patched beam for static and dynamic responses.
They also performed finite element simulations for PZT-5H actuated beam and compare the responses to those of the analytical model. 
Rakotondrabe et al. \cite{rakotondrabe2006plurilinear} used a piezoelectric unimorph system for controlling bending deformations of an active beam.
They have conducted experiments on the piezoelectric unimorph beam and compensated vibration of piezoelectric unimorph.
Using piezoelectric in controlling vibration in the active beam is shown to be effective to reduce unwanted vibration.
The analytical results are shown to be in good agreement with their experiments \cite{rakotondrabe2008hysteresis}.  
Utilizing a piezoelectric composite beam as an actuator for adaptive structures has been considered by Correia et al. \cite{franco2000modelling}.
They treated the adaptive beam as a multilayered composite and offered general properties of the beam.
They compared their work with experiments and other results from literature.
They also offered a comprehensive review on different piezoelectric models and experiments used in multifunctional composites.
Suleman and Venkayya \cite{suleman1995simple} examined polymeric piezo electric beams under bending. 
They used finite element for analyzing bending of a PVDF beam.
They considered PVDF bimorph beam for sensing and also actuation.
They showed that their finite element analyses is in a good agreement with the sensing and actuation experiments using PVDF beam. \\
  

\subsubsection{Active Trusses} 
Truss systems consist of relatively slender members connected by joints.
The joints connect the translations and allow the members to rotate with respect to each others. 
Recent advances in active materials, such as shape memory and electro-active materials, allow for generating autonomous compliant structures,
in which the structure can change their shape from one configuration to another configuration. 
This shape change is controlled by utilizing the multifunctional properties of active materials. 
Employing embedded piezoelectric actuation to actuate truss like system has been proposed by Moored et al. \cite{moored2011analytical}.
They compared this type of actuation with other strategies such as tension wires for tensegrity truss like structures that aims to mimic flapping of an artificial pectoral fin.
They proposed that their approach costs minimal power consumption and shows the simple design of a high performance tensegrity-based artificial pectoral fin.
There have been several studies on understanding the geometry of these truss systems actuated by shape memory and piezoelectric materials.
Sofla et al. \cite{sofla2009shape} studies morphing hinged truss structures, where they used shape memory wires in order to activate their structure. 
They presented an experimental study of their prototype for different configurations. 
They have used a truss structure made of tetrahedrons truss elements. 
Macareno et al. \cite{macareno2008fem} considered a linear truss made of 3D tetrahedral units for Variable Geometry Trusses (VGTs). 
They discussed manufacturing of prototype and the actuation methodology of the VGTs.
They showed that their VGT configuration made of five-module is quite capable for positioning purposes. 
They have presented the detailed design of their joints and used finite element analyses in order to support the design and motion control of VGTs. 
Aguirrebeitia et al. \cite{aguirrebeitia2009metamodeling} applied optimization technique for VGTs to modify their trusses. 
Aviles et al. \cite{Avis2000233} investigated the position problems in the open-loop variable geometry trusses. 
An optimization scheme is designed \cite{Avis2000233} in order to minimize the actuator's displacement and consequently its energy. 
This method has been applied to different truss architectures in specific to modular tetrahedral linear truss. 
Recently, Bilbao et al. \cite{Bilbao2012134} have considered dynamic analyses of their previously designed modular configuration.


\section{Motivation and Research Objectives} 
Experimental studies show that the electro - mechanical response of piezoelectric materials is time- (and rate-) dependent.
This time dependency is observed even for ceramics based piezoelectric materials, 
subjected to electric fields and mechanical loadings. 
The electro-mechanical response of these materials also depends strongly upon the applied electric fields and mechanical stresses. 
There have been constitutive models derived for predicting nonlinear electro-mechanical response of piezoelectric ceramics based materials.
These models are focusing on time (rate) independent.
Modeling the time-dependent electro-mechanical response of piezoelectric and ferroelectric ceramics based materials is still limited. 
The goal in this work is to capture the nonlinear and time-dependent response of electro active materials, 
through formulating constitutive material models and providing solutions methods.
The solution methods allows for analyzing coupled nonlinear electro-mechanical and time dependent response in active structures.
The analyses can support design of electro-active structures.

The objectives of this study are to:

1)	Formulate nonlinear time-dependent electro-mechanical constitutive models of piezoelectric ceramics with small deformation gradients taking into account polarization switching and minor hysteretic behaviors.

2)	Develop FE methods for analyzing the nonlinear and time-dependent electro-mechanical responses for piezoelectric ceramics.

3)	Perform large scale analyses of active structures undergoing various histories of electro-mechanical stimuli. 
