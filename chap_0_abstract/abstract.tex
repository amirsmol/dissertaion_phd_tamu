%%%%%%%%%%%%%%%%%%%%%%%%%%%%%%%%%%%%%%%%%%%%%%%%%%%
%
%  New template code for TAMU Theses and Dissertations starting Fall 2012.  
%  For more info about this template or the 
%  TAMU LaTeX User's Group, see http://www.howdy.me/.
%
%  Author: Wendy Lynn Turner 
%	 Version 1.0 
%  Last updated 8/5/2012
%
%%%%%%%%%%%%%%%%%%%%%%%%%%%%%%%%%%%%%%%%%%%%%%%%%%%
%%%%%%%%%%%%%%%%%%%%%%%%%%%%%%%%%%%%%%%%%%%%%%%%%%%%%%%%%%%%%%%%%%%%%
%%                           ABSTRACT 
%%%%%%%%%%%%%%%%%%%%%%%%%%%%%%%%%%%%%%%%%%%%%%%%%%%%%%%%%%%%%%%%%%%%%

\chapter*{ABSTRACT}
\addcontentsline{toc}{chapter}{ABSTRACT} % Needs to be set to part, so the TOC doesnt add 'CHAPTER ' prefix in the TOC.

\pagestyle{plain} % No headers, just page numbers
\pagenumbering{roman} % Roman numerals
\setcounter{page}{2}

\indent 
This study presents nonlinear and time-dependent analyses of ferroelectric materials and structures.
Phenomenological constitutive models are considered for simulating macroscopic responses of materials undergoing various histories of electro-mechanical inputs,
such as, polarization switching under large electric fields and minor hysteretic loops under relatively low electric fields.  
The nonlinearity is due to large electric field inputs.
The constitutive models are implemented at each material (Gaussian) points within continuum finite elements. 
Nonlinear finite element method is used for obtaining solutions to electro-mechanical boundary value problems (BVPs). 
At the end, several examples of structural analyses are presented.
These examples illustrate the nonlinear and time-dependent electro-mechanical responses of smart structures.


\pagebreak{}
